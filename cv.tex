%!TEX TS-program = xelatex
\documentclass[]{friggeri-cv}
\usepackage{afterpage}
\usepackage{hyperref}
\usepackage{color}
\usepackage{xcolor}
\usepackage{setspace}
\hypersetup{
    pdftitle={},
    pdfauthor={},
    pdfsubject={},
    pdfkeywords={},
    colorlinks=false,       % no lik border color
   allbordercolors=white    % white border color for all
}
\addbibresource{bibliography.bib}
\RequirePackage{xcolor}
\definecolor{pblue}{HTML}{0395DE}

\begin{document}
\header{}{\ \ \ \ \ \ \ \ Altamir Bitencourt Junior}
{}
\\
% Fake text to add separator      
\fcolorbox{white}{gray}{\parbox{\dimexpr\textwidth-2\fboxsep-2\fboxrule}{%
.....
}}

% In the aside, each new line forces a line break
\begin{aside}
    \section{Dados Pessoais}
    \singlespacing
    Data de Nasc: \\ 23/01/1998
    Brasileiro
    ~
    \section{Endereço}
    \singlespacing
    Rua Jacó Mariano, nº 197, Forquilhinhas - São José,
    Santa Catarina
    ~
    \section{Contato}
    \singlespacing
    +55 48 99849-4688
    +55 48 99627-1120
    Skype: juninho-jagua\\
    \href{https://goo.gl/5WVW6W}{Linkedin: altamirbitenocurtjr}
    ~
    \section{E-mail}
    \singlespacing
    \href{mailto:altamir-ju@hotmail.com}{\textbf{altamir-ju@\\hotmail.com}}
    ~
    \section{Bitbucket}
    \singlespacing
    \href{https://bitbucket.org/abitencourtjunior/}{abitencourtjunior}
    ~
    ~
    \section{Linguagens de Programação}
    \singlespacing
    % \textbf{C/C++}\includegraphics[scale=0.40]{img/3stars.png}
    \textbf{Python}\includegraphics[scale=0.40]{img/4stars.png}
    \textbf{Java}\includegraphics[scale=0.40]{img/4stars.png}
    \textbf{JSF}\includegraphics[scale=0.40]{img/3stars.png}
    \textbf{JPA}\includegraphics[scale=0.40]{img/3stars.png}
    \textbf{Hibernate}\includegraphics[scale=0.40]{img/3stars.png}
    \textbf{SQL}\includegraphics[scale=0.40]{img/4stars.png}
    \textbf{PDI}\includegraphics[scale=0.40]{img/3stars.png}
    \textbf{Javascript}\includegraphics[scale=0.40]{img/3stars.png}
    \textbf{Node}\includegraphics[scale=0.40]{img/3stars.png}
    \textbf{HTML/CSS}\includegraphics[scale=0.40]{img/3stars.png}
    \textbf{LaTeX}\includegraphics[scale=0.40]{img/4stars.png}
    \textbf{Git}\includegraphics[scale=0.40]{img/4stars.png}
    \textbf{SVN}\includegraphics[scale=0.40]{img/2stars.png}
    ~
    \section{Idiomas}
    \singlespacing
    \textbf{Inglês}\includegraphics[scale=0.40]{img/2stars.png}
    \textbf{Espanhol}\includegraphics[scale=0.40]{img/3stars.png}
    
\end{aside}

\singlespacing
\section{Experiência}

\begin{entrylist}

  \entry
    {Atual \\Abril/19}
    {Desenvolvedor Junior Java}
    {Brasil Direct Software, Florianópolis, Santa Catarina}
    {Atividades: Desenvolvimento e manutenção de sistemas na linguagem Java Web utilizando JSF, Hibernate, JPA, banco de dados relacionais(Mysql, Postgres, Dbase, SQL Server), elaboração de relatório com JasperReports, elaboração de manuais de desenvolvimento e boas práticas.\\}
  
  \entry
    {Setembro/18 \\ Março/19}
    {Analista de Testes e Suporte TI}
    {EMS Ventura, Florianópolis, Santa Catarina}
    {Atividades: Análise reporte de cliente de bugs de produção, criação de cenários de testes, desenvolvimento de manuais de novas funcionalidades, testes de interface utilizando a ferramenta TestCafé e realização de consultas SQL para geração de relatórios para clientes e em cenários de teste.}
    \\
    
  \entry
    {Agosto/18 \\Fevereiro/18}
    {Analista de Suporte Técnico Júnior}
    {Intelbras, São José, Santa Catarina}
    {Atividades: Promover atendimento ao cliente e suporte técnico via telefone, chat e e-mail no segmento de Segurança Eletrônica, promovendo procedimentos, configurações, aplicações em relação aos produtos e desenvolvimento de ferramentas para auxiliar o suporte técnico.\\}
  
%   \entry
%     {Fevereiro/18 \\Julho/17}
%     {Estagiário - Suporte Técnico CFTV}
%     {Intelbras, São José, Santa Catarina}
%     {Atividades: Promover atendimento ao cliente e suporte técnico via chat no segmento de Segurança Eletrônica, promovendo procedimentos, configurações e aplicações em relação aos produtos.\\}
  
%   \entry
%     {02/17 - 06/17}
%     {Atendente de Lanchonete}
%     {RM Campinas - Cia Sucos, São José, Santa Catarina}
%     {Atividades: Atendimento a clientes e realização de rotinas administrativas.\\}
    
%   \entry
%     {12/09 - 06/09}
%     {Jovem Aprendiz - TI}
%     {Terex Latim America, Cotia, São Paulo}
%     {Atividades: Promover suporte técnico NVL 1 para colaboradores, realizando manutenção em microcomputadores,  criação perfis de usuários no Active Directory, criação de scripts para execução de serviços, configurações de redes e documentar ativos e hosts da rede.\\}
    
\end{entrylist}

\section{Educação}
\begin{entrylist}
  \entry
    {2017 - 2021}
    {Bacharelado em Ciência da Computação - 6ª Fase}
    {Faculdade IES/FASC, São José}
    {Em Andamento, formação em 2020.}
    
    \entry
    {2013 - 2015}
    {Ensino Médio - Completo}
    {Escola Estadual Sidrônia Nunes Pires, Cotia/SP}

  \entry
    {2014 - 2015}
    {Técnico em Redes de Computadores}
    {Centro Paula Souza - ETEC, Cotia/SP}
    {Ementa: Suporte de Microcomputadores e Servidores, Gerenciamento de redes locais, aplicação em ambientes com servidores Linux/UNIX as ferramentas de DHCP, DNS, AD, PROXY, FIREWALL e na plataforma Windows AD, IMPRESSÃO, ARQUIVOS, DHCP, DHCPV6, DNS, PROXY, WEB.}
    
  \entry
    {2012 - 2012}
    {Aprendizagem Industrial em Suporte de Microcomputadores e Redes Locais}
    {Senai/SC , Tubarão}
    {Ementa: Suporte de Microcomputadores, configuração redes locais de baixa complexidade.}
  
\end{entrylist}

\section{Certificações}
\begin{entrylist}

    \entry
    {200 horas \\Concluído}
    {Curso de Extensão EAD - IFSP/Sertãozinho}
    {Programação de Computadores e Dispositivos Móveis}
    {\emph{A ementa do curso visa programação, onde foram abordados linguagens estruturadas como C, linguagens orientadas a objetos JAVA e C++, banco de dados relacional SQL - Mysql e SQLite e desenvolvimento de aplicativos móveis na plataforma Android utilizando a linguagem Java.}}
    \\
    \entry
    {12 horas \\Concluído}
    {Alura Cursos}
    {NODE.JS PARTE 1: INOVANDO COM JAVASCRIPT NO BACKEND}
    {\emph{O objetivo do curso visa o ensino de construir uma sistema com um CRUD utilizando a linguagem PHP e banco de dados Mysql.}}
    \\
    \entry
    {20 horas \\Concluído}
    {Alura Cursos}
    {Curso JavaScript: Programando na linguagem da web}
    {\emph{Criar uma aplicação web para fins nutricionais, cálculo de IMC e gordura, POO, Expressões Ternárias e outros. }}
    \\
    \entry
    {52 horas \\Andamento}
    {Alura Cursos}
    {CARREIRA JAVA}
    {\emph{
    Java parte 1: PRIMEIROS PASSOS - 8 horas\\
    Java parte 2: INTRODUÇÃO À ORIENTAÇÃO A OBJETOS - 8 horas \\
    Java parte 3: ENTENDENDO HERANÇA E INTERFACE - 16 horas \\
    Java parte 4: ENTENDENDO EXCEÇÕES - 12 horas\\
    Java e JDBC : TRABALHANDO COM UM BANCO DE DADOS - 8 horas\\
    Java parte 5: Pacotes e java.lang - Andamento \\
    Java parte 6: Conhecendo o java.util - Andamento\\
    Java parte 7: Trabalhando com java.io - Andamento}}
    \\
    \entry
    {20 horas \\Concluído}
    {Alura Cursos}
    {JASPER REPORTS: RELATÓRIOS COM JAVA}
    {\emph{Curso com intuito de demonstrar as melhores práticas para usar a ferramenta Jasper Reports utilizando direto da ferramenta IReport e JasperStudio e também implentação diretamente de uma aplicação Java Web. }}
    \\
    \entry
    {20 horas \\Concluído}
    {Alura Cursos}
    {MYSQL}
    {\emph{
    MYSQL I: INICIANDO SUAS CONSULTAS - 8 Horas \\
    INTRODUÇÃO AO SQL: MANIPULE DADOS COM MYSQL - 12 Horas}}
    \\
    \entry
    {14 horas \\Andamento}
    {Alura Cursos}
    {PENTAHO DATA INTEGRATION - PDI}
    {\emph{
    MODELOS DE ETL: PENTAHO DATA INTEGRATION - 14 Horas \\
    Transformação com ETL: Pentaho Data Integration - Andamento}}
    \\
    \entry
    {1 hora \\Concluído}
    {Alura Cursos}
    {LEAN STARTUP}
    {\emph{
    LEAN STARTUP: PRIMEIROS PASSOS DA SUA STARTUP ENXUTA - 1 hora}}
    \\
    \entry
    {12 horas \\Concluído}
    {Alura Cursos}
    {PYTHON 3}
    {\emph{
    PYTHON 3 PARTE 1: INTRODUÇÃO À NOVA VERSÃO DA LINGUAGEM}}
    \\
    % \entry
    % {32 horas \\Andamento}
    % {Alura Cursos}
    % {Curso HTML5 e CSS3 I: Suas primeiras páginas da Web}
    % {\emph{Criar páginas desde o zero utilizando marcações HTML e aplicando estilos CSS para estilizar as mesmas. }}
    % \\
    % \entry
    % {12 horas \\Concluído}
    % {Alura Cursos}
    % {PHP E MYSQL I: FUNDAMENTOS PARA CRIAR UM SISTEMA NA WEB}
    % {\emph{O objetivo do curso visa o ensino de construir uma sistema com um CRUD utilizando a linguagem PHP e banco de dados Mysql.}}
    % \entry
    % {04/2016}
    % {Segurança da Informação}
    % {Escola Virtual - Fundação Bradesco}
    % {\emph{A finalidade do curso se relaciona aos princípios de segurança da informação e formas de como proteger os dados.}}
    % \\
    % \entry
    % {12/2012}
    % {Cisco Networking Academy}
    % {Cisco - IT Essentials : PC Hardware and Software}
    % {\emph{O curso era ministrado em paralelo com as aulas na aprendizagem industrial, de modo que os assuntos abordados tinham grande base nesta certificação para manutenção de computadores, redes locais e outros.}}

\end{entrylist}

\section{Informações Adicionais}
\begin{entrylist}
    \entry    {06/2018}
    {Participação na 2ª Edição do Hackathon Payment Shift 2018}
    {- Unisociesc - Florianópolis}
    {\emph{ Durante a participação do evento, criamos um aplicativo para Android para pagamentos, a fim de otimizar e melhorar a prestação de serviços e venda para postos de gasolina, no qual nos proporcionou o 3º Lugar no evento. O aplicativo desenvolvido e ficou entitulado como Pitstop. \\
    O desenvolvimento do aplicativo foi realizado utilizando a IDE do Android Studio junto com a linguagem Java.}}
    % \\
    % \entry{12/2012}
    % {Carta de Recomendação - Cisco Networking Academy}
    % { - Cisco - IT Essentials : PC Hardware and Software}
    % {Este documento foi parte da premiação para alunos com as melhores notas no decorrer do curso.}

\end{entrylist}

\end{document}
